\section{Notifikasi Matematika}
Sebagai aplikasi editor pengolah dokumen, LATEX memiliki kemampuan yang mampu menghasilkan dokumen berisi notasi-notasi matematis. Agar dapat menghasilkan dokumen yang berisikan notasi-notasi matematis maka kita harus berada dalam \textit{Mathematics Environtment}.

\section{Penulisan Notasi Matematika}
Pada latex kita dapat menuliskan suatu notasi matematika yang cukup panjang dalam suatu paragraf baru. Penulisan Notasi Matematika dalam suatu paragraf dapat dilihat pada listing \ref{lst:notasi_paragraf}.
\lstinputlisting[caption=Notasi Matematika Dalam Paragraf,label={lst:notasi_paragraf}]{src/3/notasi1.tex}

\section{Font Dalam Notasi Matematika}
Ada beberapa perintah pada yang dapat digunakan untuk mengubah jenis font notasi matematis dalam latex. Beberapa perintah tersebut dapat kita lihat pada listing \ref{lst:fontmath}.
\lstinputlisting[caption=Jenis Font Matematis,label={lst:fontmath}]{src/3/font.tex}

Hasil output : 

$\mathrm{x y z}$

$\mathsf{x y z}$

$\mathtt{x y z}$

$\mathit{x y z}$

$\mathbf{x y z}$

\section{Perumusan Menggunakan Supersripts dan Subscripts}
Penulisan \textit{Supserscripts} dan \textit{Subscripts} biasanya digunakan untuk membuat sebuah rumus dengan menghasilkan pangkat diatas dan pangkat dibawah pada suatu rumus. Cara penulisan penggunaan ini adalah dengan menggunakan perintah \textbf{sp} dan perintah \textbf{sb}. Untuk contoh penerapan perintah \textit{Supserscripts} dan \textit{Subscripts} dapat kita lihat pada listing \ref{lst:sp1}.

\lstinputlisting[caption=Penggunaan Supersripts dan Subscripts,label={lst:sp1}]{src/3/sp1.tex}

Hasil output :

\begin{displaymath}
y = x\sb{1}\sp{2} + x\sb{2}\sp{2}
\end{displaymath}

Atau kita juga dapat menggunakan perintah lain seperti pada listing \ref{lst:sp2}.

\lstinputlisting[caption=Perintah Pada Superscripts dan Subscripts,label={lst:sp2}]{src/3/sp2.tex}

Hasil output :

\begin{displaymath}
f(x) = e^{x_1}
\end{displaymath}