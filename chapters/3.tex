\section{Membuat Rumus dengan LaTex}
Sebagai aplikasi editor pengolah dokumen, LATEX memiliki kemampuan yang mampu menghasilkan dokumen berisi notasi-notasi matematis. Agar dapat menghasilkan dokumen yang berisikan notasi-notasi matematis maka kita harus berada dalam \textit{Mathematics Environtment}.

\section{Penulisan Notasi Matematika}
Pada latex kita dapat menuliskan suatu notasi matematika yang cukup panjang dalam suatu paragraf baru. Penulisan Notasi Matematika dalam suatu paragraf dapat dilihat pada listing \ref{lst:notasi_paragraf}.
\lstinputlisting[caption=Notasi Matematika Dalam Paragraf,label={lst:notasi_paragraf}]{src/3/notasi1.tex}

\section{Font Dalam Notasi Matematika}
Ada beberapa perintah pada yang dapat digunakan untuk mengubah jenis font notasi matematis dalam latex. Beberapa perintah tersebut dapat kita lihat pada listing \ref{lst:fontmath}.
\lstinputlisting[caption=Jenis Font Matematis,label={lst:fontmath}]{src/3/font.tex}

Hasil output : 

$\mathrm{x y z}$

$\mathsf{x y z}$

$\mathtt{x y z}$

$\mathit{x y z}$

$\mathbf{x y z}$

\section{Rumus Dasar}
Rumus dasar ini terdiri dari 3 notasi yaitu penjumlahan, pengurangan, dan perkalian. Contoh kode untuk rumus dasar bisa dilihat pada listing \ref{lst:rumus_dasar}.
\lstinputlisting[caption=Penggunaan Rumus Dasar,label={lst:rumus_dasar}]{src/1/rumus_dasar.tex}
Hasil output:

$$ a+b$$

$$ a-b$$

$$ a \times b$$

\subsection{Rumus Pecahan}
Rumus pecahan yang dimaksud adalah notasi per pada pembagian. Contoh kode untuk rumus pecahan bisa dilihat pada listing \ref{lst:rumus_pecahan}.
\lstinputlisting[caption=Penggunaan Rumus Pecahan,label={lst:rumus_pecahan}]{src/1/rumus_pecahan.tex}
Hasil output:

$$ a/b$$

$$ \frac {a}{b}$$

\subsection{Rumus Akar}
Rumus akar dapat dilihat pada listing \ref{lst:rumus_akar}.
\ref{lst:rumus_akar}.
\lstinputlisting[caption=Penggunaan Rumus Akar,label={lst:rumus_akar}]{src/1/rumus_akar.tex}
Hasil output:

$$ \sqrt[a]{b}$$

$$ \sqrt{\sqrt{a}}$$

\section{Perumusan Menggunakan Superscripts dan Subscripts}
Penulisan \textit{Supserscripts} dan \textit{Subscripts} biasanya digunakan untuk membuat sebuah rumus dengan menghasilkan pangkat diatas dan pangkat dibawah pada suatu rumus. Cara penulisan penggunaan ini adalah dengan menggunakan perintah \textbf{sp} dan perintah \textbf{sb}. Untuk contoh penerapan perintah \textit{Supserscripts} dan \textit{Subscripts} dapat kita lihat pada listing \ref{lst:sp1}.

\lstinputlisting[caption=Penggunaan Supersripts dan Subscripts,label={lst:sp1}]{src/3/sp1.tex}

Hasil output :

\begin{displaymath}
y = x\sb{1}\sp{2} + x\sb{2}\sp{2}
\end{displaymath}

Atau kita juga dapat menggunakan perintah lain seperti pada listing \ref{lst:sp2}.

\lstinputlisting[caption=Perintah Pada Superscripts dan Subscripts,label={lst:sp2}]{src/3/sp2.tex}

Hasil output :

\begin{displaymath}
f(x) = e^{x_1}
\end{displaymath}

\section {Perumusan Array dan Matriks}
Dalam LaTex, kita dapat menuliskan rumus sebuah array pada environment \textbf{tabular}. Perintah untuk membuat array dan matriks dapat kita lihat pada listing \ref{lst:array}.
\lstinputlisting[caption=Penulisan Array atau Matriks,label={lst:array}]{src/3/array.tex}

Hasil output :

\begin{displaymath}
\left (
\begin{array}{rrr}
0 & 55 & 23 \\
34 & -83 & 68 \end{array}
\right )
\end{displaymath}

Ada beberapa hal yang perlu kita ketahui dalam penulisan rumus array atau matriks ini :
\begin{itemize}
\item Penulisan array memiliki kesamaan seperti saat membuat format tabel
\item Perintah \textbf{"rrr"} berfungsi untuk menentukan posisi dari masing-masing komponen matriks tersebut
\item Tanda kurung kurawal "( )" berfungsi untuk mendefinisikan bagian kurung buka dan kurung tutup pada sebuah matriks  
\end{itemize}



