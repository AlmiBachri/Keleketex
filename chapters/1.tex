\section{Pembagian bab}
Pembagian bab di latex menggunakan perintah 


\section{Format Cetak}
Hal paling mendasar antara lain cetak tebal, miring dan gari bawah. Cetak tebal menggunakan perintah 

\section{Tanda petik}
Tanda petik di Latex menggunakan petik miring dan petik satu. Petik miring biasanya berada pada sebelah angka satu di keyboard dan diakhiri petik satu.

\begin{lstlisting}[caption=Contoh kalimat dalam tanda petik di Latex,label={lst:tandapetik}]
`kalimat dalam tanda petik'
\end{lstlisting}

\section{Penomoran}
Penomoran di latex menggunakan perintah 

\section{Karakter Khusus}

Sebuah dokumen LATEX memiliki struktur yang dicirikan dengan blok yang diapit oleh pasangan perintah (begin) dan (end). Untuk menyatakan jenis dokumen yang akan diolah, setiap dokumen harus dimulai dengan perintah


Dalam perintah diatas,\textit{class} dapat diganti oleh article, report, book,atau slides untuk menuliskan artikel,laporan,buku,atau transparansi untuk seminar. Sedangkan pada bagian \textit{option} dapat dituliskan satu atau beberapa pilihan berikut:10pt, 11pt, 12pt untuk menyatakan ukuran font utama yang digunakan didalam dokumen paper, letter paper menyatakan ukuran kertas yang digunakan titlepage. No titlepage untuk menyatakan apakah halaman judul akan dibuat terpisah dari badan dokumen atau tidak twocolumn untuk menampilkan dokumen dalam bentuk dua kolom twoside. Oneside untuk menyatakan apakah dokumen akan dicetak pada satu sisi atau dua sisi dari kertas.


\section{Kode Program}
Kode program menggunakan 

\section {PErintah LAtex}
a. Spasi dalam Latex
Ada perintah khusus untuk membuat spasi dengan panjang tertentu baik secara horizontal maupun vertikal, yaitu :

Jika ingin membuat jarak dengan panjang tertentu antara 2 baris, dapat menggunakan tanda garis miring di akhir baris. Dan juga dapat menentukan sendiri panjang baris kosong dengan menggunakan perintah seperti contoh berikut ini :

baris 1



baris 2
Dengan perintah ini, Latex akan mengosongkan baris-baris sepanjang 2 cm. Tanpa menggunakan perintah ini untuk membuat spasi dalam teks dokumen, Latex akan tetap menganggapnya 1 spasi.

Jika ingin membuat spasi sejauh beberapa centimeter antara 2 kata dibutuhkan perintah sebagai berikut :

Dengan perintah ini, Latex akan membuat spasi sejauh 2 centimeter.

Jadi, secara umum aturan yang dapat dipakai adalah akhiri paragraf dengan tanda garis miring dan berikan 1 baris kosong antara tiap-tiap paragraf dan 1 spasi kosong antara masing-masing kata.

Ada beberapa macam ukuran font dalam format Latex, untuk menggunakan ukuran yang ada dalam format Latex itu ada beberapa caranya yaitu sebagai berikut:




untuk ukuran font jenis perintah yang dapat kita gunakan adalah:
Tiny
Scriptsize
Footnotesize
Small
Normal
Large
Larger
Largest
Huge
Huger

