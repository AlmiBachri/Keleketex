\section{Pembagian bab}
Pembagian bab di latex menggunakan perintah \textit{section}, \textit{subsection}, \textit{subsubsection} dan \textit{subsubsubsection}.


\section{Format Cetak}
Hal paling mendasar antara lain cetak tebal, miring dan gari bawah. Cetak tebal menggunakan perintah \textit{textbf},cetak miring menggunakan perintah \textit{textit} dan garis bawah menggunakan perintah \textit{underline}.

\section{Tanda petik}
Tanda petik di Latex menggunakan petik miring dan petik satu. Petik miring biasanya berada pada sebelah angka satu di keyboard dan diakhiri petik satu. Ingat fungsi tanda petik hanya untuk melakukan quote atau pengutipan langsung. Untuk istilah bahasa inggris gunakan miring.

\begin{lstlisting}[caption=Contoh kalimat dalam tanda petik di Latex,label={lst:tandapetik}]
`kalimat dalam tanda petik'
\end{lstlisting}

\section{Penomoran}
Penomoran di latex menggunakan perintah \textit{enumerate} sedangkan untuk poin menggunakan \textit{itemize}.

\section{Karakter Khusus}
Beberapa karakter yang tidak bisa langsung digunakan seperti tanda ampersand


\section{Kode Program}
Kode program menggunakan \textit{lstlisting}. Jangan lupa parameter \textit{caption} dan \textit{label} senantiasa ditulis.
\lstinputlisting[caption=Menambahkan kode program,label={lst:kodeprogram}]{src/1/lstlisting.tex}

\section{Rumus}

\section{Tabel}


penulisan judul

dalam penulisan judul dalam format latex di letakkan pada awal document, untuk cara penulisan nya sebagai berikut:

garis miring document class kurung kurawal a4papper, ukuran yang di inginkan tutup kurawal lalu report

garis miring begin buka kurawal document tutup kurawal

garis miring begin buka kurawal judul document tutup kurawal

garis miring autor buka kurawal nama penulis tutup kurawal 

garis miring date buka kurawal tanggal pembuatan tutup kurawal
 
garis miring maketitle

garis miring and buka kurawal document tutup kurawal 