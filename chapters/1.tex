\section{Pembagian bab}
Pembagian bab di latex menggunakan perintah \textit{section}, \textit{subsection}, \textit{subsubsection} dan \textit{subsubsubsection}.


\section{Format Cetak}
Hal paling mendasar antara lain cetak tebal, miring dan gari bawah. Cetak tebal menggunakan perintah \textit{textbf},cetak miring menggunakan perintah \textit{textit} dan garis bawah menggunakan perintah \textit{underline}.

\section{Tanda petik}
Tanda petik di Latex menggunakan petik miring dan petik satu. Petik miring biasanya berada pada sebelah angka satu di keyboard dan diakhiri petik satu.

\begin{lstlisting}[caption=Contoh kalimat dalam tanda petik di Latex,label={lst:tandapetik}]
`kalimat dalam tanda petik'
\end{lstlisting}

\section{Penomoran}
Penomoran di latex menggunakan perintah \textit{enumerate} sedangkan untuk poin menggunakan \textit{itemize}.

\section{Karakter Khusus}

Sebuah dokumen LATEX memiliki struktur yang dicirikan dengan blok yang diapit oleh pasangan perintah \begin dan \end. Untuk menyatakan jenis dokumen yang akan diolah, setiap dokumen harus dimulai dengan perintah \documentclass{…}
Jenis dokumen yang akan diolah ditentukan oleh perintah pertama dalam bentuk \documentclass[option]{class}

Dalam perintah diatas,“class”dapat diganti oleh article, report, book,atau slides untuk menuliskan artikel,laporan,buku,atau transparansi untuk seminar. Sedangkan pada bagian“option” dapat dituliskan satu atau beberapa pilihan berikut:10pt, 11pt, 12pt untuk menyatakan ukuran font utama yang digunakan didalam dokumen paper, letter paper menyatakan ukuran kertas yang digunakan titlepage. No titlepage untuk menyatakan apakah halaman judul akan dibuat terpisah dari badan dokumen atau tidak twocolumn untuk menampilkan dokumen dalam bentuk dua kolom twoside. Oneside untuk menyatakan apakah dokumen akan dicetak pada satu sisi atau dua sisi dari kertas. 

\section{Kode Program}
Kode program menggunakan \textit{lstlisting}. Jangan lupa parameter \textit{caption} dan \textit{label} senantiasa ditulis.
\lstinputlisting[caption=Menambahkan kode program,label={lst:kodeprogram}]{src/1/lstlisting.tex}
