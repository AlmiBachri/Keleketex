\section{Pembagian bab}
Secara default pembagian bab pada latex menggunakan perintah \textit{section}, \textit{subsection}, \textit{subsubsection} dan \textit{subsubsubsection}. Untuk mengatur kedalaman suatu dokumen pada bab bab tertentu, kita dapat menggunakan perintah berikut ini pada bagian Preamble :



setcounter.secnumdepth


setcounter.tocdepth

Opsi yang digunakan pada syntax secnumdepth pada perintah verbcounter= seperti perintah diatas, berarti Anda telah merubah kedalaman bab yang Anda perbaharui sampai dengan level 5 yaitu section -- subsection -- subsubsection -- paragraph -- subparagraph. Sedangkan pada perintah dari opsi tocdepth berfungsi untuk membuat table of contents atau menampilkan kedalaman bab sampai dengan level 5, namun jika tidak di setel maka pada bagian level 3 kebawah tidak akan dapat ditampilkan pada bagian toc. 


\section{Format Cetak}
Hal paling mendasar antara lain cetak tebal, miring dan gari bawah. Cetak tebal menggunakan perintah \textit{textbf},cetak miring menggunakan perintah \textit{textit} dan garis bawah menggunakan perintah \textit{underline}.

\section{Tanda petik}
Tanda petik di Latex menggunakan petik miring dan petik satu. Petik miring biasanya berada pada sebelah angka satu di keyboard dan diakhiri petik satu. Ingat fungsi tanda petik hanya untuk melakukan quote atau pengutipan langsung. Untuk istilah bahasa inggris gunakan miring.

\begin{lstlisting}[caption=Contoh kalimat dalam tanda petik di Latex,label={lst:tandapetik}]
`kalimat dalam tanda petik'
\end{lstlisting}

\section{Penomoran}
Penomoran di latex menggunakan perintah \textit{enumerate} sedangkan untuk poin menggunakan \textit{itemize}.

\section{Karakter Khusus}
Beberapa karakter yang tidak bisa langsung digunakan seperti tanda ampersand


\section{Kode Program}
Kode program menggunakan \textit{lstlisting}. Jangan lupa parameter \textit{caption} dan \textit{label} senantiasa ditulis.
\lstinputlisting[caption=Menambahkan kode program,label={lst:kodeprogram}]{src/1/lstlisting.tex}

\section{Rumus}

\section{Tabel}

